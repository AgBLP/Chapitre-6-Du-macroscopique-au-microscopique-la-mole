\modeCorrection

\enTeteChap{6}{De l'espèce chimique à l'entité.}
\renewcommand{\arraystretch}{1}

\begin{center}
    \begin{LARGE}
        \'{E}valuation diagnostique : \\ \textbf{Modélisation de la matière à l'échelle microscopique}
    \end{LARGE}
\end{center}


\begin{large}
    Répondez aux questions suivantes ou entourer \underline{la} bonne réponse :
\end{large}

\question{Classer les objets qu'on peut trouver à \underline{l'échelle microscopique} ou à \underline{l'échelle macroscopique} parmi les suivants : un \textbf{grain de sable}, un \textbf{atome de silicium}, un \textbf{cachet de paracétamol}, la \textbf{molécule de paracétamol}, \textbf{des cristaux de sel} ; \textbf{les ions sodium Na$^+$}.\\
\begin{center}
\begin{tabular}{|C{0.45}|C{0.45}|}
\hline
     \cellcolor{orange!25}\'{E}chelle macroscopique & \cellcolor{orange!25}\'{E}chelle macroscopique  \\
     \hline
     \rule[-7pt]{0pt}{50pt} & \rule[-7pt]{0pt}{50pt}  \\
     \hline
\end{tabular}
\end{center}}{\begin{center}
\begin{tabular}{|C{0.45}|C{0.45}|}
\hline
     \cellcolor{orange!25}\'{E}chelle macroscopique & \cellcolor{orange!25}\'{E}chelle microscopique  \\
     \hline
     un grain de sable ; un cachet de paracétamol ; des cristaux de sel & un atome de silicium ; la molécule de paracétamol ; les ions sodium Na$^{+}$\\
     \hline
\end{tabular}
\end{center}}{0}

\question{Indiquer le nombre d'atomes de carbone et d'hydrogène dans le méthane de formule chimique \chemform{CH_4}.\newline \texteTrouMultiLignes{}{0}}{Il y a un atome de carbone et quatre atomes d'hydrogène.}{0}
\\
\question{Lorsqu'on dissout du sel \chemform{NaCl} dans l'eau, l'ion Na$^{+}$ formé est : 
\begin{align*}
    a.& ~\text{électriquement neutre} & b.& ~\text{un anion} & c.& ~\text{un cation}
\end{align*}}{c.~\text{un cation}}{0}
\\
\question{\'{E}crire la formule chimique du composé ionique formé par les ions \chemform{Mg^{2+}} et \chemform{Cl^-}. \newline \texteTrouMultiLignes{}{0}}{\chemform{MgCl_2}}{0}
\\
\question{Rappeler la constitution un atome. Indiquer le signe de la charge électrique de chaque élément de l'atome.\newline \texteTrouMultiLignes{}{1}}{Un atome est constitué d'un noyau (chargé positivement) et d'un ou plusieurs électrons (chargés négativement).}{0}
\\
\question{La taille d'un atome est de l'ordre de : \begin{align*}
    a.&~ 10^{-3}~\text{m} & b.&~ 10^{3}~\text{m} & c.& ~10^{-10}~\text{m}
\end{align*}}{c.~$10^{-10}~\text{m}$}{0}