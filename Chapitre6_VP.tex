\modeCorrection

\renewcommand{\thesubsection}{\textcolor{red}{\Roman{section}.\arabic{subsection}}}
\renewcommand{\thesubsubsection}{\textcolor{red}{\Roman{section}.\arabic{subsection}.\alph{subsubsection}}}

\setcounter{section}{0}
\sndEnTeteCoursQuatre

\begin{mdframed}[style=titr, leftmargin=60pt, rightmargin=60pt, innertopmargin=7pt, innerbottommargin=7pt, innerrightmargin=8pt, innerleftmargin=8pt]

\begin{center}
\large{\textbf{Chapitre 6 : De l'espèce chimique à l'entité : la mole}}
\end{center}
\end{mdframed}


\begin{tcolorbox}[colback=blue!5!white,colframe=blue!75!black,title=Mots clés du chapitre :]

\end{tcolorbox}


\section{Constitution de la matière}
\subsection{\`{A} l'échelle microscopique}
\begin{tcolorbox}[colback=red!5!white,colframe=red!75!black,title=\textbf{Propriété de la propagation :}]
La lumière se propage de façon \textcolor{red}{rectiligne} (en ligne droite) dans un milieu \textcolor{red}{transparent} (qui laisse passer la lumière) et \textcolor{red}{homogène} (dont les propriétés ne changent pas dans l'espace).
\end{tcolorbox}

\subsection{\`{A} l'échelle macroscopique}
La vitesse, aussi appelée \textcolor{red}{célérité} de la lumière, est c = 299 792 458~m$\cdot$s$^{-1}$. Aucun objet matériel ne peut aller plus vite que la lumière.
\begin{tcolorbox}[colback=red!5!white,colframe=red!75!black,title=\textbf{Propriété de la vitesse de la lumière :}]
On retiendra que dans le vide et dans l'air :
\begin{empheq}[box=\fbox]{equation*}
    c_{vide}\simeq c_{air}=3\times10^8~\text{m$\cdot$s$^{-1}$}
\end{empheq}
\end{tcolorbox}

\begin{tcolorbox}
[colback=green!5!white,colframe=green!75!black,title=\textbf{Indice optique d'un milieu homogène :}]
Un milieu transparent et homogène est caractérisé par son \textcolor{red}{indice optique} (ou \textcolor{red}{indice de réfraction}), noté $n$ défini par :
\begin{equation*}
    n_{\text{milieu}} = \frac{c}{v_{\text{milieu}}}
\end{equation*}
avec :
\begin{itemize}
    \item c la célérité de la lumière dans le vide, 
    \item $v_{\text{milieu}}$ la vitesse de la lumière dans le milieu.
\end{itemize}
\importantbox{L'indice optique d'un milieu est toujours supérieur à 1 : $n_{milieu}>1$.}
\end{tcolorbox}

\section{La quantité de matière}
\begin{Large}
    \ding{43}
\end{Large}
Voir TP 9 : Les lois de la réfraction et de la réflexion.
\subsection{Détermination de la masse d'une entité}
\begin{tcolorbox}[colback=red!5!white,colframe=red!75!black,title=\textbf{Règle sur les molécules et les ions polyatomiques : }]
\begin{itemize}[label=\textbullet, font=\large]
    \item La masse des molécules se calcule en faisant la somme des masses de chacun des atomes la constituant. Exemple pour le glucose de formule chimique \chemform{C_6H12O6} : 
        \begin{equation*}
            m(\text{\chemform{C_6H12O6}} = 6\times m(C) + 12\times m(H) + 6\times m(O)
        \end{equation*}
    \item C'est la même règle pour déterminer la masse d'un ion polyatomique. Exemple pour l'ion sulfate de formule chimique \chemform{SO_4}$^{-2}$ :
        \begin{equation*}
            m(\text{\chemform{SO_4}$^{-2}$}) = 1\times m(S) + 4\times{O}
        \end{equation*}
\end{itemize}

\end{tcolorbox}

\subsection{Dénombrer les entités : la mole}
On peut calculer le nombre d'entités chimiques présents dans un échantillon de masse $m_{ech}$ à partir de la masse des atomes présents dans cette espèce : 
\begin{itemize}[label=\textbullet, font=\large]
    \item 1 entité $\rightarrow$ $m$
    \item N entités $\rightarrow$ $m_{ech}$
    \item d'où $N\frac{m_{ech}}{m}$
\end{itemize}
Mais c'est compliqué puisque le calcul donne N très très grand ! Plutôt que de compter en entités, on décide de les rassembler par paquet appelé : \textcolor{red}{mole}.
\begin{tcolorbox}[colback=green!5!white,colframe=green!75!black,title=\textbf{La mole :}]
Une \textcolor{red}{mole} d'une espèce chimique contient un extrêmement grand d'entités chimiques la constituant. En appelant $N$ le nombre d'entités chimiques et $n$ le nombre de moles d'une espèce chimique, il existe : 
\begin{empheq}[box=\fbox]{equation*}
    N = n\times\mathrm{N_A}
\end{empheq}
entités chimiques dans $n$ moles d'espèce chimique. $\mathrm{N_A}$ est appelé \textcolor{red}{constante d'Avogadro} et vaut : 
\begin{empheq}[box=\fbox]{equation*}
    \mathrm{N_A} = 6,022\times10^{23}\text{mol$^{-1}$}
\end{empheq}
\end{tcolorbox}

\begin{Large}
    \ding{45}
\end{Large}\textbf{Exercice 14, 19}
\section{Modélisation de l'\oe il par une lentille}
\begin{Large}
    \ding{43}
\end{Large}
Voir Activité : L'\oe il, un instrument remarquable.

\subsection{Modélisation}
Tout comme une lentille, l'\oe il permet de visualiser un \textcolor{red}{objet} à une certaine distance en faisant converger les rayons lumineux issus de cet objet pour en produire une \textcolor{red}{image} sur la rétine :
\begin{center}
    %\includegraphics[scale=0.5]{Images/Modele_oeil.PNG}
\end{center}

\subsection{Détermination graphique d'une image par une lentille convergente}

\begin{tcolorbox}[colback=green!5!white,colframe=green!75!black,title=\textbf{Lentille convergente :}]
\begin{multicols}{2}
    %\includegraphics[scale=0.7]{Images/Lentille.PNG}
    
    Une lentille est caractérisée par son \textcolor{red}{centre optique} O par lequel passe l'axe optique $\Delta$, son \textcolor{red}{foyer image} F' et \textcolor{red}{son foyer objet} F.
\end{multicols}
\end{tcolorbox}
%\newpage


\begin{tcolorbox}[colback=red!5!white,colframe=red!75!black,title=\textbf{Construction d'une image par une lentille :}]
\begin{itemize}
    \item Le rayon passant par le centre optique O \textbf{n'est pas dévié} ;
    \item Les rayons arrivant parallèlement à l'axe optique $\Delta$ \textbf{ressortent sur le foyer image F'} ;
    \item Les rayons passant par le foyer objet F \textbf{ressortent parallèle à l'axe optique $\Delta$}.
\end{itemize}
\end{tcolorbox}



\subsection{Grandissement}
Pour un objet de taille AB, on obtient une image A'B'. 
\begin{tcolorbox}[colback=green!5!white,colframe=green!75!black,title=\textbf{Grandissement :}]
On définit le \textcolor{red}{grandissement} $\gamma$ d'une lentille comme le rapport entre la \textcolor{red}{hauteur algébrique} de l'image $\overline{\text{A'B'}}$ sur la \textcolor{red}{hauteur algébrique} de l'objet $\overline{\text{AB}}$ :
\begin{empheq}[box=\fbox]{equation*}
    \gamma = \frac{\overline{A'B'}}{\overline{AB}}
\end{empheq}
\begin{itemize}
    \item Si $\gamma$ est négatif, alors l'image est renversée ;
    \item Si $\abs{\gamma}$ est plus petit que 1, alors l'image est plus petite que l'objet.
\end{itemize}
\end{tcolorbox}

\begin{mdframed}[style=autreexo]
\textbf{\bsc{Exercice de cours} - Grandissement}\\
Sur la construction précédente, déterminer $\gamma$.
\end{mdframed}

\begin{Large}
    \ding{45}
\end{Large}\textbf{Exercice 25, Appareil photo}